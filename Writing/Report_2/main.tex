\documentclass{article}
\usepackage[utf8]{inputenc}
\usepackage{amsmath}
\usepackage{graphicx}
%\usepackage{enumerate}
\usepackage{amsfonts}
\usepackage{natbib}
\usepackage{url} % not crucial - just used below for the URL 
\usepackage{cleveref}
\usepackage{float}
\usepackage{subfigure}
%\usepackage{lineno,hyperref} 
\usepackage{xcolor}

%\pdfminorversion=4
% NOTE: To produce blinded version, replace "0" with "1" below.
\newcommand{\blind}{0}
\let\oldref\ref
\renewcommand{\ref}[1]{(\oldref{#1})}
\DeclareMathOperator*{\argmin}{argmin}
\DeclareMathOperator*{\argmax}{argmax}


\newtheorem{prop}{Proposition}
\newtheorem{assumption}{Assumption}
\newtheorem{thm}{Theorem}
\newtheorem{proof}{Proof}


\title{Simulation and Timeline}
%author{xmengju }
%\date{March 2021}

\begin{document}
\maketitle


\section{Robust TFBoost: Simulation}
\subsection{Data generation}
\begin{itemize}
    \item We generated data sets $D = \{(x_i, y_i), i = 1,..., N\}$, consisting of a predictor $x_i\in\mathcal{L}_2$ and a scalar response $y_i$ that follow the model: 
 \begin{equation} \label{eq:gen}
y_i = r(x_i) + \rho \epsilon_i,
\end{equation}
where the errors $\epsilon_i$ are i.i.d, $r$ is the regression function,  and  $\rho > 0$ is a constant that controls the signal-to-noise ratio (SNR): 
$$\text{SNR} = \frac{\text{Var}(r(X))}{\text{Var}(\rho\epsilon)}.$$
\item To sample the functional predictors $x_i$, we considered the model:

\begin{equation} \label{eq:xmodel}
    x_i(t) = \mu(t) + \sum_{p=1}^4 \sqrt{\lambda_j}\xi_{ij}\phi_j(t),
\end{equation}
where $\mu(t) = 2\text{sin}(t\pi) \text{exp}(1-t)$, $\lambda_1 = 0.8, \lambda_2 = 0.3, \lambda_3 = 0.2$, and $\lambda_4 = 0.1$,  $\xi_{ij}\sim N(0,1) $,  and $\phi_j$ are the first four eigenfunctions of the ``Mattern'' covariance function $\gamma(s,t)$ with parameters $\rho = 3, \sigma = 1, \nu = 1/3$: 
 $$\gamma(s,t) = C\left(\frac{\sqrt{2\nu}|s-t|}{\rho}\right), \ C(u) = \frac{\sigma^2 2^{1-\nu}}{\Gamma(\nu)} u^{\nu} K_{\nu}(u),$$
  where $\Gamma(.)$ is the Gamma function and $K_{\nu}$ is the modified Bessel function of the second kind. For each subject $i$, we evaluate $x_i$ on a dense and regular grid $t_1,..., t_{100}$ equally spaced in $\mathcal{I} = [0,1]$. 
\item We considered five regression functions:
\begin{itemize}

\item[- ]  $r_1(X) =  \int_{\mathcal{I}} \left (\text{sin} \left(\frac{3}{2} \pi t \right) +  \text{sin} \left(\frac{1}{2} \pi t \right)\right)X(t)dt,$
\item[- ]  $r_2(X) = (\xi_1 + \xi_2)^{1/3},$ where  $\xi_1 = \int_{\mathcal{I}} (X(t) - \mu(t))\psi_1(t) dt$ and $\xi_2 = \int_{\mathcal{I}} (X(t) - \mu(t))\psi_2(t) dt$ are projections onto the first two FPCs ($\psi_1$ and $\psi_2$) of $X$ with mean $\mu(t) = E(X(t))$, 
\item[- ]  $r_3(X) = 5\text{exp}\left (- \frac{1}{2}\left| \int_{\mathcal{I}} x(t)\log(|x(t)|)dt \right| \right),$
\item[- ] 
$r_4(X) = 5\text{sigmoid}\left(\int_{\mathcal{I}}X(t)^2 \text{sin}(2\pi t) dt \right),$ where  $\text{sigmoid}(u) = 1/(1+ \text{exp}(-u))$, and
\item[- ] 
$r_5(X) = 5 \left( \sqrt{\left|\int_{\mathcal{I}_1} \text{cos}(2\pi t^2) X(t) dt \right|} + \sqrt{\left|\int_{\mathcal{I}_2} \text{sin}(X(t)) dt \right|} \right), $ where  $\mathcal{I}_1 = [0,0.5]$ and $\mathcal{I}_2 = (0.5,1]$. 
\end{itemize}

\item For clean data ($C_0$), we generated $\epsilon_i$ in \ref{eq:gen} from $N(0,1)$ and selected $\rho$ that corresponds to SNR = 5. 

For contaminated data, we sampled 10\% training samples as outliers and let the set of their indices be $I_{\text{o}}$. The outliers belong to  one of the five types introduced below. For $j \in I_{\text{o}}$, 
\begin{itemize}
    \item[- ] $C_1$: \textit{Shape outliers}
    
    \vspace{1ex}
    In \ref{eq:gen}, $\epsilon_j \sim N(10, 0.25)$ \\
    In \ref{eq:xmodel},   $\xi_{j,2} \sim N(10, 0.25)$ and the other parameters stay the same.  
       \vspace{1ex}
    \item[- ] $C_2$: \textit{Magnitude outliers}     \vspace{1ex}
    
    $x_{j} = 2 \tilde{x}_{j}, y_{j} =  4 \tilde{y}_{j},$ where $(\tilde{x}_j, \tilde{y}_j)$ were generated as clean data.
       \vspace{1ex}
    \item[- ] $C_3$: \textit{Point-type measurement error outliers} 
    
       \vspace{1ex}
   Randomly sample 10  points form $t_1,..., t_{100}$ and denote them as $t_{j,o_1},..., t_{j,o_{10}}$. For $k = 1,..., 10$,   
    $$x_{j}(t_{j,o_k}) = \tilde{x}_j(t_{j,o_k}) + \eta_{j,o_k},$$ where $\eta_{j,o_k} \sim 0.5 N(10, 0.25) + 0.5 N(-10, 0.25)$, $y_j = \tilde{y}_j$, and 
 $(\tilde{x}_j, \tilde{y}_j)$ were generated as clean data. 
    \vspace{1ex}
    \item[- ] $C_4$: \textit{Interval-type measurement error outliers} 
    
       \vspace{1ex}
     Randomly sample one interval from intervals $[t_1,...,t_{10}]$, ...,$[t_{91},...,t_{100}]$,   and denote the interval as $t_{j,o},..., t_{j,o+9}$
     
     For $k = 0,..., 9$,   
    $$x_{j}(t_{j,o + k}) = \tilde{x}_j(t_{j,o + k}) + \eta_{j,o+k},$$ where $\eta_{j,o + k} \sim  N(10, 0.25)$, $y_j = \tilde{y}_j$, and 
 $(\tilde{x}_j, \tilde{y}_j)$ were generated as clean data. 
    \vspace{1ex}
 
    \item[- ] $C_5$: \textit{Pure vertical outliers} 
       \vspace{1ex}
   $$\epsilon_{j} \sim N(10, 0.25)$$
\end{itemize}
\end{itemize}


\subsection{Visualize the outliers}

\begin{figure}[H]
    \centering
    \includegraphics[scale = 0.8]{visualize_outliers.pdf}
\end{figure}

\subsection{Model comparison}
For each setting, we used 100 independently generated datasets and compared the performance of the following methods: 

\begin{itemize}
 \setlength\itemsep{0.1em}
\item  \texttt{TFBoost(L2)}:  tree-based functional boosting with L2 loss
\item  \texttt{TFBoost(LAD)}:  tree-based functional boosting with LAD loss
\item  \texttt{TFBoost(RR)}:  tree-based functional boosting modified to follow the framework of RRBoost
\item \texttt{FPPR}: functional projection pursuit regression \citep{ferraty2013functional},
\item \texttt{FGAM}: functional generalized additive models \citep{mclean2014functional}, 
\item \texttt{MFLM}: Sieve M-estimator for a semi-functional linear model \cite{huang2015sieve}
\item \texttt{RFSIR}: robust functional sliced inverse regression \citep{wang2017robust}
\item \texttt{RFPLM}: robust estimation for semi-functional linear regression models \citep{boente2020robust}
\end{itemize}

\subsection{Results}
% latex table generated in R 4.0.5 by xtable 1.8-4 package
% Thu Jul 15 22:49:42 2021
\renewcommand{\arraystretch}{1.5}
\addtolength{\tabcolsep}{-3pt}    
\begin{table}[H]
\centering
\footnotesize
\begin{tabular}{rllllll}
  \hline
 & $C_0$ & $C_1$ & $C_2$ & $C_3$ & $C_4$ & $C_5$ \\ 
  \hline
TFBoost(L2) & 0.144 (0.006) & 0.151 (0.009) & 0.206 (0.030) & 0.144 (0.006) & 0.146 (0.006) & 1.690 (0.271) \\ 
  TFBoost(LAD) & 0.151 (0.010) & 0.202 (0.027) & 0.205 (0.030) & 0.150 (0.008) & 0.151 (0.008) & \textbf{0.158} (0.012) \\ 
  TFBoost(RR) & 0.162 (0.017) & 0.193 (0.136) & 0.215 (0.110) & 0.158 (0.014) & 0.157 (0.012) & 0.159 (0.015) \\ 
  FPPR & 0.137 (0.007) & 0.202 (0.076) & 0.164 (0.050) & 0.137 (0.007) & 0.149 (0.013) & 1.845 (0.517) \\ 
  FGAM & \textbf{0.130} (0.005) & \textbf{0.143} (0.008) & \textbf{0.153} (0.016) & \textbf{0.130} (0.005) & \textbf{0.133} (0.005) & 1.205 (0.080) \\ 
  RFPLM & \textbf{0.130} (0.006) & \textbf{0.130} (0.006) & \textbf{0.130} (0.006) & \textbf{0.130} (0.006) & \textbf{0.131} (0.006) & \textbf{0.130} (0.006) \\ 
  MFLM & \textbf{0.129} (0.006) & 0.761 (0.054) & 0.269 (0.033) & \textbf{0.130} (0.006) & 0.138 (0.006) & 0.166 (0.014) \\ 
  RFSIR & 0.137 (0.008) & 0.145 (0.014) & 0.157 (0.025) & 0.138 (0.007) & 0.142 (0.006) & 1.727 (0.587) \\ 
   \hline
\end{tabular}
\caption{Summary statistics of test errors for data generated from $r_1$; displayed in the form of mean (sd).} 
\end{table}
% latex table generated in R 4.0.5 by xtable 1.8-4 package
% Thu Jul 15 22:49:54 2021
\begin{table}[H]
\centering
\footnotesize
\begin{tabular}{rllllll}
  \hline
 & $C_0$ & $C_1$ & $C_2$ & $C_3$ & $C_4$ & $C_5$ \\ 
  \hline
TFBoost(L2) & \textbf{0.181} (0.008) & \textbf{0.193} (0.010) & 0.223 (0.023) & \textbf{0.183} (0.009) & \textbf{0.184} (0.010) & 1.789 (0.347) \\ 
  TFBoost(LAD) & 0.186 (0.010) & 0.257 (0.056) & 0.208 (0.020) & 0.188 (0.011) & \textbf{0.188} (0.011) & \textbf{0.195} (0.011) \\ 
  TFBoost(RR) & 0.196 (0.014) & 0.225 (0.063) & \textbf{0.198} (0.019) & 0.202 (0.021) & 0.198 (0.023) & \textbf{0.203} (0.020) \\ 
  FPPR & \textbf{0.181} (0.009) & 0.347 (0.133) & 0.288 (0.058) & \textbf{0.183} (0.011) & 0.196 (0.024) & 1.886 (0.545) \\ 
  FGAM & 0.226 (0.012) & 0.243 (0.015) & 0.276 (0.027) & 0.233 (0.013) & 0.233 (0.012) & 1.343 (0.094) \\ 
  RFPLM & 0.286 (0.014) & 0.286 (0.014) & 0.290 (0.016) & 0.287 (0.014) & 0.288 (0.017) & 0.286 (0.014) \\ 
  MFLM & 0.285 (0.014) & 2.032 (0.099) & 0.325 (0.028) & 0.389 (0.023) & 0.626 (0.032) & 0.344 (0.024) \\ 
  RFSIR & 0.183 (0.009) & \textbf{0.218} (0.061) & \textbf{0.202} (0.015) & 0.185 (0.010) & 0.193 (0.013) & 1.551 (0.569) \\ 
   \hline
\end{tabular}
\caption{Summary statistics of test errors for data generated from $r_2$; displayed in the form of mean (sd).} 
\end{table}
% latex table generated in R 4.0.5 by xtable 1.8-4 package
% Thu Jul 15 22:50:06 2021
\begin{table}[H]
\centering
\footnotesize
\begin{tabular}{rllllll}
  \hline
 & $C_0$ & $C_1$ & $C_2$ & $C_3$ & $C_4$ & $C_5$ \\ 
  \hline
TFBoost(L2) & \textbf{0.305} (0.016) & \textbf{0.314} (0.020) & 0.518 (0.090) & \textbf{0.306} (0.015) & \textbf{0.308} (0.016) & 1.968 (0.360) \\ 
  TFBoost(LAD) & 0.319 (0.019) & 0.382 (0.036) & 0.383 (0.049) & 0.317 (0.018) & 0.318 (0.016) & \textbf{0.326} (0.021) \\ 
  TFBoost(RR) & 0.333 (0.032) & 0.370 (0.059) & \textbf{0.337} (0.041) & 0.337 (0.040) & 0.328 (0.028) & \textbf{0.335} (0.027) \\ 
  FPPR & \textbf{0.303} (0.018) & 0.446 (0.105) & 0.606 (0.360) & 0.313 (0.022) & 0.318 (0.022) & 1.845 (0.453) \\ 
  FGAM & 0.319 (0.017) & 0.331 (0.017) & 0.442 (0.061) & 0.321 (0.016) & 0.319 (0.017) & 1.445 (0.112) \\ 
  RFPLM & 0.380 (0.018) & 0.379 (0.019) & 0.381 (0.018) & 0.379 (0.019) & 0.382 (0.019) & 0.379 (0.019) \\ 
  MFLM & 0.377 (0.018) & 1.365 (0.080) & 0.485 (0.046) & 0.886 (0.063) & 2.165 (0.129) & 0.445 (0.028) \\ 
  RFSIR & 0.310 (0.019) & \textbf{0.310} (0.019) & \textbf{0.337} (0.030) & \textbf{0.311} (0.020) & \textbf{0.311} (0.017) & 1.825 (0.677) \\ 
   \hline
\end{tabular}
\caption{Summary statistics of test errors for data generated from $r_3$; displayed in the form of mean (sd).} 
\end{table}
% latex table generated in R 4.0.5 by xtable 1.8-4 package
% Thu Jul 15 22:50:18 2021
\begin{table}[H]
\centering
\footnotesize

\begin{tabular}{rllllll}
  \hline
 & $C_0$ & $C_1$ & $C_2$ & $C_3$ & $C_4$ & $C_5$ \\ 
  \hline
TFBoost(L2) & \textbf{0.321} (0.015) & \textbf{0.333} (0.015) & 0.681 (0.322) & \textbf{0.324} (0.014) & \textbf{0.328} (0.014) & 2.037 (0.267) \\ 
  TFBoost(LAD) & \textbf{0.338} (0.017) & 0.404 (0.026) & 0.552 (0.161) & \textbf{0.340} (0.014) & \textbf{0.345} (0.017) & \textbf{0.361} (0.026) \\ 
  TFBoost(RR) & 0.347 (0.036) & 0.490 (0.273) & 0.591 (0.667) & 0.365 (0.036) & 0.374 (0.048) & \textbf{0.360} (0.040) \\ 
  FPPR & 0.362 (0.029) & 0.417 (0.045) & 0.538 (0.271) & 0.384 (0.040) & 0.415 (0.043) & 1.960 (0.386) \\ 
  FGAM & 0.408 (0.019) & 0.417 (0.018) & \textbf{0.491} (0.054) & 0.415 (0.017) & 0.411 (0.019) & 1.659 (0.151) \\ 
  RFPLM & 0.544 (0.032) & 0.544 (0.031) & 0.544 (0.032) & 0.555 (0.040) & 0.561 (0.039) & 0.543 (0.032) \\ 
  MFLM & 0.538 (0.030) & 0.544 (0.032) & 0.826 (0.104) & 0.645 (0.037) & 0.827 (0.048) & 0.630 (0.048) \\ 
  RFSIR & 0.341 (0.020) & \textbf{0.363} (0.021) & \textbf{0.421} (0.154) & 0.348 (0.018) & 0.354 (0.024) & 2.415 (0.512) \\ 
   \hline
\end{tabular}
\caption{Summary statistics of test errors for data generated from $r_4$; displayed in the form of mean (sd).} 
\end{table}
% latex table generated in R 4.0.5 by xtable 1.8-4 package
% Thu Jul 15 22:50:30 2021
\begin{table}[H]
\centering
\footnotesize

\begin{tabular}{rllllll}
  \hline
 & $C_0$ & $C_1$ & $C_2$ & $C_3$ & $C_4$ & $C_5$ \\ 
  \hline
TFBoost(L2) & \textbf{0.583} (0.032) & \textbf{0.628} (0.045) & 1.725 (0.678) & \textbf{0.593} (0.033) & \textbf{0.590} (0.036) & 2.322 (0.278) \\ 
  TFBoost(LAD) & 0.622 (0.034) & 0.694 (0.055) & 1.292 (0.315) & \textbf{0.633} (0.039) & \textbf{0.634} (0.030) & \textbf{0.677} (0.066) \\ 
  TFBoost(RR) & 0.694 (0.092) & 0.869 (0.307) & 1.280 (1.596) & 0.703 (0.083) & 0.723 (0.084) & \textbf{0.686} (0.077) \\ 
  FPPR & \textbf{0.608} (0.057) & 0.718 (0.175) & 0.967 (0.911) & 0.638 (0.049) & 0.673 (0.073) & 2.364 (0.482) \\ 
  FGAM & 0.610 (0.041) & \textbf{0.670} (0.070) & \textbf{0.776} (0.100) & 0.643 (0.046) & 0.641 (0.048) & 1.909 (0.127) \\ 
  RFPLM & 0.891 (0.045) & 1.045 (0.078) & 0.890 (0.045) & 0.889 (0.046) & 0.895 (0.047) & 0.888 (0.045) \\ 
  MFLM & 0.881 (0.039) & 1.125 (0.067) & 1.698 (0.189) & 1.421 (0.080) & 2.527 (0.118) & 0.998 (0.052) \\ 
  RFSIR & 0.677 (0.052) & 0.690 (0.063) & \textbf{0.821} (0.183) & 0.672 (0.052) & 0.650 (0.053) & 2.379 (0.518) \\ 
   \hline
\end{tabular}
\caption{Summary statistics of test errors for data generated from $r_5$; displayed in the form of mean (sd).} 
\end{table}
\renewcommand{\arraystretch}{1.5}
\addtolength{\tabcolsep}{3pt}    



\subsection{Timeline}
\begin{itemize}
    \item 2021/07: 
    \begin{itemize}
        \item  TFBoost: revise paper (submit?)
        \item Robust TFBoost: simulation 
        \item  thesis: draft the background chapter 
    \end{itemize}
   \item  2021/08:  \begin{itemize}
   \item  TFBoost: submit paper and package
        \item   thesis: draft the background,  RRBoost and TFBoost chapters 
        \item  Robust TFBoost: simulation and real example
        \item record JSM presentation
    \end{itemize}
    \item  2021/09:
    \begin{itemize}
    \item thesis: draft Robust TFBoost chapter
    \item Sparse TFBoost: simulation 
    \end{itemize}
   \item  2021/09:
    \begin{itemize}
    \item thesis: draft robust TFBoost, Sparse TFBoost chapters 
    \item Sparse TFBoost: simulation and real example
    \end{itemize}
    \item  2021/10:
    \begin{itemize}
    \item thesis: draft Sparse TFBoost chapter, conclusion and future work
    \end{itemize}
\item  2021/11:
    \begin{itemize}
    \item thesis: first draft complete, start revising 
    \end{itemize}
\item  2021/12 (end of year):     
\begin{itemize}
    \item thesis: second draft 
    \end{itemize}
\item Before  2022/04:
\begin{itemize}
    \item thesis defence 
    \end{itemize}
\end{itemize}

\bibliographystyle{apalike}
\bibliography{reference}
\end{document}
